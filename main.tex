\documentclass[a4paper,12pt]{book}
\usepackage[utf8]{inputenc}

\usepackage{amsmath, enumitem, bm, xcolor}
\usepackage[colorlinks, linkcolor = red, citecolor = black, filecolor = black, urlcolor = blue]{hyperref} 
% \usepackage{mathptmx}
\usepackage{setspace}
%\singlespacing
\onehalfspacing

\renewcommand{\t}[1]{\text{#1}}
\renewcommand{\d}{\text{d}}
\renewcommand{\b}[1]{\boldsymbol{#1}}
\newcommand{\B}[1]{\mathbf{#1}}
\newcommand{\D}[1]{{#1}^\dagger}
\newcommand{\fD}[1]{{#1}^{ \phantom{\dagger}}}
\renewcommand{\P}[1]{{#1}^\prime}
\newcommand{\fP}[1]{{#1}^{\phantom{\prime}}}

% transpose
\usepackage{relsize}
\newcommand{\tp}[1]{{#1}^{\mathrm t}}
% \!\: = 1mu distance; \!\; = 2mu (\, = 3mu)
\newcommand{\itp}[1]{{#1}^{\!\:\scalebox{0.55}[1.0]{\( - \)}1 \mathrm t}}

% comments:
\definecolor{dblue}{RGB}{31,119,180}
\newcommand{\REM}[1]{\textcolor{dblue}{{\bf REM: #1 }}}

\usepackage{blindtext}

% https://en.wikibooks.org/wiki/LaTeX/Title_Creation
\title{
    Thermal Conductivities of Strongly Anharmonic Compounds from First Principles
}
\author{
    Florian Knoop
    \\ \and
    First Supervisor: Prof. Dr. Matthias Scheffler
    \and
    Second Supervisor: Dr. Christian Carbogno
    \\ \and
    Fritz-Haber-Institut der Max-Planck-Gesellschaft, Berlin
}
\date{2020}

\begin{document}
\maketitle

\chapter{Introduction}
One of the major challenges mankind faces in the 21th century is the responsible and sustainable handling of the earth's natural resources.  In fact, most energy 
today is lost during the transformation of raw energy sources to usable 
power. To date, there is no fuel based heat engine that exceeds an 
efficiency of 50\,\% and often it is even worse~\cite{eia}. 
%\emph{Increasing} power efficiency is thus tantamount 
%to \emph{decreasing} waste heat emission or finding clever ways to \emph{reuse} it. 
Since gas- and aircraft-turbines
are essentially Carnot engines, their efficiency and core power are directly related to the 
combustion temperature~\cite{Clarke2012,Perepezko2009}.
This has been utilized during the past 30 years by developing 
ceramics with high thermal resistivity that are nowadays applied as 
\emph{thermal barrier coatings} on turbine airfoils in heat engines~\cite{Clarke2003}. 
A thermal barrier coating serves as a thin, extremely heat insulating layer and thus allows to 
operate a turbine at
higher temperatures, thereby increasing its efficiency.

A complementary strategy %to improve efficiency 
is to recycle waste heat where it occurs.
%, especially in 
%applications where its emission is not diminishable or avoidable. 
One way to do so is to use the
\emph{thermoelectric effect} to 
generate electric power from temperature gradients~\cite{Tritt2}. The main obstacle 
preventing mass operation though is the very limited conversion rate (figure of merit) of even the 
most 
advanced thermoelectric materials known to date, especially when taking production costs 
into account~\cite{Nolas2001}. A prerequisite to finding better thermoelectrics is the 
determination of their thermal conductivity, which is inversely proportional to the figure of merit.

\chapter{Theory and Methods}
Lorem ipsum

\chapter{Screening}
Lorem ipsum

\chapter{Results}
Lorem ipsum

\end{document}
