\documentclass[a4paper,12pt]{book}
\usepackage[utf8]{inputenc}

\usepackage{amsmath, enumitem, bm, xcolor}
\usepackage[colorlinks, linkcolor = red, citecolor = black, filecolor = black, urlcolor = blue]{hyperref} 

% Quotes
\usepackage{epigraph}
\setlength\epigraphwidth{12cm}
\setlength\epigraphrule{0pt}

% \usepackage{mathptmx}
\usepackage{setspace}
%\singlespacing
\onehalfspacing

\renewcommand{\t}[1]{\text{#1}}
\renewcommand{\d}{\text{d}}
\renewcommand{\b}[1]{\boldsymbol{#1}}
\newcommand{\B}[1]{\mathbf{#1}}
\newcommand{\D}[1]{{#1}^\dagger}
\newcommand{\fD}[1]{{#1}^{ \phantom{\dagger}}}
\renewcommand{\P}[1]{{#1}^\prime}
\newcommand{\fP}[1]{{#1}^{\phantom{\prime}}}
\newcommand{\im}{{\rm i}}

% transpose
\usepackage{relsize}
\newcommand{\tp}[1]{{#1}^{\mathrm t}}
% \!\: = 1mu distance; \!\; = 2mu (\, = 3mu)
\newcommand{\itp}[1]{{#1}^{\!\:\scalebox{0.55}[1.0]{\( - \)}1 \mathrm t}}

% comments:
\definecolor{dblue}{RGB}{31,119,180}
\newcommand{\REM}[1]{\textcolor{blue}{{\bf #1}}}
\newcommand{\CITE}[1]{\textcolor{blue}{{\bf [#1]}}}
% \newcommand{\FK}[1]{\textcolor{blue}{{\bf #1 }}}

\usepackage{blindtext}

% https://en.wikibooks.org/wiki/LaTeX/Title_Creation
\title{
    Thermal Conductivities of Strongly Anharmonic Compounds from First Principles
}
\author{
    Florian Knoop
    \\ \and
    First Supervisor: Prof. Dr. Matthias Scheffler
    \and
    Second Supervisor: Dr. Christian Carbogno
    \\ \and
    Fritz-Haber-Institut der Max-Planck-Gesellschaft, Berlin
}
\date{2020}

\begin{document}
\maketitle
\tableofcontents

\chapter{Introduction}
\epigraph{\singlespacing \it ``Die Zeit des unbedenklichen Wirtschaftens mit den Energiequellen und Stofflagern, die uns die Natur zur Verfügung gestellt hat, wird wahrscheinlich schon für unsere Kinder nur noch die Bedeutung einer vergangenen Wirtschaftsepoche haben.''}{W. Schottky, 1929}
% [ ] Why thermal conductivity?
% [ ] Why novel thermal insulators?
% [ ] Why computational materials science?
One of the major challenges humankind faces in the 21th century is the responsible and sustainable handling of the earth's natural resources~\cite{Schottky1929}.  Yet, most energy today is lost as waste heat during the transformation of raw energy sources to usable power. To date, there is no fuel based heat engine that exceeds an efficiency of 50\,\% and often it is even worse~\cite{eia}. 
Since gas- and aircraft-turbines are essentially Carnot engines, their efficiency and core power are directly related to the combustion temperature~\cite{Clarke2012,Perepezko2009}. This has been utilized during the past 30 years by developing 
ceramics with high thermal resistivity that are nowadays applied as \emph{thermal barrier coatings} on turbine airfoils in heat engines~\cite{Clarke2003}. A thermal barrier coating serves as a thin, extremely heat insulating layer and thus allows to operate a turbine at higher temperatures, thereby increasing its efficiency.

A complementary strategy is to recycle waste heat where it occurs. One way to do so is to use the
\emph{thermoelectric effect} to  generate electric power from temperature gradients~\cite{Snyder2008}. The main obstacle preventing mass operation though is the limited conversion rate (figure of merit) $zT$ of even the most advanced thermoelectric materials known to date. To make matters worse, these materials often contain heavy metals and are toxic, and their manufacturing process is difficult and expensive~\cite{Nolas2001}. Recent advancements in the field, such as the discovery of a high thermoelectric figure of merit in the lead-free material Tin Selenide~\cite{Zhao2014}, offer hope that novel materials with significant figure of merit can be found that are non-toxic, easy and cheap to produce, and consist of abundant elements.

A key physical property of both thermal barrier coatings (TBCs) and thermoelectrics is their thermal conductivity $\kappa$. In the case of thermoelectrics, the figure of merit is inversely proportional to $\kappa$~\cite{Nolas2001}:
\begin{align}
	zT = \frac{S^2 \sigma_{\rm el}}{\kappa} T~,
	\label{eq:zT}
\end{align} 
where $T$ dentoes the temperature, $S$ the Seebeck coefficient, and $\sigma_{\rm el}$ is the electrical conductivity.
A prerequisite to finding better thermoelectrics or TBCs therefore is to find materials which are thermally insulating. These are typically non-metals, since the free electrons in metals are good heat carriers, and most of the known thermoelectrics are thermally insulating inorganic semiconductors~\cite[p.\,15]{Nolas2001}.

% [ ] say sth. about microstructuring?

Despite the technological needs, systematic knowledge of thermal conductivities in inorganic compounds is scarce. A renowned database like Springer Materials only lists thermal conductivities for about 200~of these compounds~\cite{SpringerMaterials}, which is partially due to the fact that accurate measurements of thermal conductivity are tricky to perform[\REM{citation!}]. As a consequence, thermal conductivity is not systematically understood beyond semi-empirical and phenomenological trends in a very limited number of simple material classes~\cite{Morelli}.
\\ \\
The aim of this work is to open a new pathway for overcoming the problem of limited data by devising a route to systematically scan material space for thermal insulators and calculate their thermal conductivities from first principles. This route is twofold: After reviewing the relevant theoretical tools necessary to simulate heat transport in thermal insulators, we describe how to assess a key physical property shared by most thermal insulators,~.i.\,e.,~\emph{anharmonicity}, without the need for explicit model building beyond the harmonic approximation.

\chapter{Theory and Methods}

\section{The Many Body Problem}
\epigraph{\singlespacing \it ``The underlying physical laws necessary
for the mathematical theory of a large part of physics and the whole of chemistry
are thus completely known, and the difficulty is only that the exact application
of these laws leads to equations much too complicated to be soluble. It therefore becomes desirable that approximate practical methods of applying quantum
mechanics should be developed, which can lead to an explanation of the main
features of complex atomic systems without too much computation.''}{P.A.M. Dirac, 1929}

In this chapter, we summarize the theoretical background of {\it ab initio} simulations.

\subsection{The Many Body Hamiltonian}

The full (non-relativistic) many body Hamiltonian in the absence of external electromagnetic fields for an otherwise arbitrary system reads
\begin{align}
    \hat{\mathrm{H}}
        = \hat{\mathrm{T}}^{\mathrm{e}}
        + \hat{\mathrm{V}}^{\mathrm{e}-\mathrm{e}}
        + \hat{\mathrm{V}}^{\mathrm{e}-\mathrm{Nuc}}
        + \hat{\mathrm{V}}^{\mathrm{Nuc}-\mathrm{Nuc}}
        + \hat{\mathrm{T}}^{\mathrm{Nuc}}~,
    \label{eq:Hamiltonian}
\end{align}
where
\begin{align}
    \hat T^{\rm e} 
        =\sum_{i} \frac{\hat{\mathbf{p}}_{i}^{2}}{2 m_{\rm e}}
    \label{eq:Te}
\end{align}
is the kinetic energy operator for electrons of mass $m_{\rm e}$ with momentum operators $\hat{\bf p}_i$, and 
\begin{align}
    \hat V^{\rm e-e}
        = \sum_{i < j} \frac{e^{2}}{\left|\hat{\b r}_{i}-\hat{\bf r}_{j}\right|}~,
    \label{eq:Ve}
\end{align}
is the Coulombic electron-electron repulsion operator with the electronic position operators $\hat{\bf r_i}$ and the elementary charge $e$. 
The Coulomb attraction between the negatively charged electrons and the positively charged nuclei reads
\begin{align}
    \hat V^{\rm e-Nuc}
        = \sum_{i, J} -\frac{Z_J e^{2}}{\left|\hat{\b r}_{i}-\hat{\bf R}_{J}\right|}~,
    \label{eq:Venuc}
\end{align}
where $Z_J$ denotes the charge number of nucleus $J$, and $\hat{\bf R}_J$ is the nuclear position operator. 
Accordingly, we define the nuclear-nuclear repulsion as
\begin{align}
    \hat V^{\rm Nuc-Nuc}
        = \sum_{I < J} \frac{Z_{I} Z_{J} e^{2}}{\left|\hat{\b R}_{I}-\hat{\bf R}_{J}\right|}~,
    \label{eq:Vnuc}
\end{align}
and the kinetic energy operator for nuclei with momentum operators $\hat{\bf P}_I$ and masses $M_I$ reads
\begin{align}
    \hat T^{\rm Nuc} 
        =\sum_{I} \frac{\hat{\mathbf{P}}_{I}^{2}}{2 M_{I}}~.
    \label{eq:Tnuc}
\end{align}

\subsection{The Born Oppenheimer Approximation}
We go over to a unitless Hamiltonian by scaling Eq.\,\eqref{eq:Hamiltonian} with the Hartree energy $E_{\rm h} = me^4 / \hbar^2 \approx 27.2\,{\rm eV}$ and expressing distances in terms of the Bohr radius $a_0 = \hbar^2 / m e^2$,~i.\,e.,~${\bf r} = a_0 \tilde{\bf r}$, where $\hbar$ denotes the Planck constant~\CITE{Czycholl}. Dropping the operator hats in the following and using $\hat{\bf p}=-\im \hbar \partial / \partial {\bf r}$ instead, we have
\begin{align}
\begin{split}
    \tilde H 
        \equiv& ~H / E_{\rm h} \\
        =& 
        - \frac{1}{2} \sum_i \frac{\partial^2}{\partial \tilde{\bf r}_i^2}
        + \sum_{i < j} \frac{1}{\left|\tilde{\b r}_{i}-\tilde{\bf r}_{j}\right|}
        - \sum_{i, J} -\frac{Z_J}{\left|\tilde{\b r}_{i}-\tilde{\bf R}_{J}\right|}
        + \sum_{I < J} \frac{Z_{I} Z_{J}}{
            \left|\tilde{\b R}_{I}-\tilde{\bf R}_{J}\right|} 
        \\
        &- \frac{1}{2} \sum_I \frac{m_{\rm e}}{M_I} \frac{\partial^2}{\partial \tilde{\bf R}_I^2}~,
    \label{eq:Hscaled}
\end{split}
\end{align}
which only depends on the charge numbers $\{Z_I\}$ and the mass ratios $\{ m_{\rm e} / M_I \}$. The relative order of magnitude of the nuclear kinetic energy is $m_{\rm e} / M_I \approx 10^{-4} - 10^{-5}$. This means that the nuclear kinetic energy can be treated as a perturbation of the remaining electronic and electron-nuclear contributions:
\begin{align}
    H   &= H^0 + T^{\rm Nuc}~, \text{ where} 
    \label{eq:H=H0+T}
    \\
    H^0 &=
        \hat{\mathrm{T}}^{\mathrm{e}}
        + \hat{\mathrm{V}}^{\mathrm{e}-\mathrm{e}}
        + \hat{\mathrm{V}}^{\mathrm{e}-\mathrm{Nuc}}
        + \hat{\mathrm{V}}^{\mathrm{Nuc}-\mathrm{Nuc}}~.
    \label{eq:H0}
\end{align}
The full (time-independent) many-body Schroedinger equation reads
\begin{align}
    H \psi ({\bf r}, {\bf R}) = E \psi ({\bf r}, {\bf R})~,
    \label{eq:Schroedinger1}
\end{align}
with ground-state eigenvalues and many-body wave functions $E$ and \mbox{$\psi ({\bf r}, {\bf R})$},
where \mbox{${\bf r} = ({\bf r}_i, \ldots, {\bf r}_{N_{\rm e}})$} denotes all electronic coordinates, and ${\bf R} = ({\bf R}_i, \ldots, {\bf R}_{N_{\rm Nuc}})$ all nuclear coordinates, respectively. According to Eq.\,\eqref{eq:H=H0+T}, we expand the wave functions in solutions to the Hamiltonian $H^0$,
\begin{align}
    H^0 \phi_l ({\bf r}, {\bf R})
        = E^0_l ({\bf R}) \phi_l ({\bf r}, {\bf R})~,
    \label{eq:Hsolution1}
\end{align}
where the dependence on $\bf R$ is parametric,~i.\,e.,~Eq.\,\eqref{eq:Hsolution1} is solved for a given nuclear configuration $\bf R$.
The full solution $\psi ({\bf r}, {\bf R})$ is given as
\begin{align}
    \psi ({\bf r}, {\bf R}) = \sum_l \chi_l ({\bf R}) \phi_l ({\bf r}, {\bf R})~.
    \label{eq:psi_expansion_phi}
\end{align}
The functions $\chi_l$ are to be determined by using Eq.\,\eqref{eq:psi_expansion_phi} in Eq.\,\eqref{eq:Schroedinger1},
\begin{align}
    (H - E) \psi ({\bf r}, {\bf R})
        & = \sum_l (H^0 + T^{\rm Nuc} - E) \chi_l ({\bf R}) \phi_l ({\bf r}, {\bf R}) \nonumber \\
        &= \sum_l (E^0_l({\bf R}) + T^{\rm Nuc} - E) \chi_l ({\bf R}) \phi_l ({\bf r}, {\bf R}) \\
    = 0 \nonumber~,
\end{align}
and integrating with $\int \d^3 r ~ \phi^\ast_m ({\bf r}, {\bf R})$, so that
\begin{align}
    \left( T^{\rm Nuc} + E^0_m({\bf R}) \right) \chi_m ({\bf R})
        + \sum_l A_{ml} ({\bf R}) \chi_l ({\bf R})
        = E \chi_m ({\bf R})~,
    \label{eq:chi1}
\end{align}
where the operator $A_{ml}$~\CITE{Czycholl},
\begin{align}
    A_{ml} ({\bf R})
        = - \sum_I \frac{\hbar^2}{2 M_I} \int \d^3 r ~ 
        &\left[ \phi^\ast_m ({\bf r}, {\bf R}) \frac{\partial^2}{\partial {\bf R}_I^2}
            \phi_l ({\bf r}, {\bf R}) \right. \nonumber \\
        &\left.
            + 2 \phi^\ast_m ({\bf r}, {\bf R}) \left(
                \frac{\partial}{\partial {\bf R}_I} \phi_l ({\bf r}, {\bf R}) \right)
            \frac{\partial}{\partial {\bf R}_I}
        \right]~,
    \label{eq:Aml}
\end{align}
describes coupling between different electronic states $(l, m)$. This term is of the order of $(m/M)^{1/4} \approx 10^{-1} - 10^{-2}$ smaller than the nuclear energy~\CITE{BornOppenheimer}. Neglecting this term is known as the \emph{Born-Oppenheimer approximation} and completely separates the dynamical evolution of electrons and nuclei. In effect, Eq.\,\eqref{eq:chi1} reduces to
\begin{align}
    \left( T^{\rm Nuc} + E^0_l({\bf R}) \right) \chi_l ({\bf R})
        = E \chi_l ({\bf R})~.
    \label{eq:chi2}
\end{align}
Solving Eq.\,\eqref{eq:chi2} is performed in two steps:
\begin{enumerate}
    \item For a given configuration $\bf R$, the electronic Schr\"odinger equation \eqref{eq:Hsolution1} is solved, yielding the energies $E^0_l({\bf R})$ which thereby parametrically depend on $\bf R$.
    \item For each electronic quantum number $l$, Eq.\,\eqref{eq:chi2} for the nuclei is solved, where the electronic energy $E^0_l({\bf R})$ defines the effective \emph{potential energy surface}.
\end{enumerate}
Pictorially, the electrons move \emph{adiabatically} with the nuclei, hence the alternative term \emph{adiabatic approximation.}

\subsection{Kohn-Sham DFT}
In the previous chapter, it was tacitly assumed that the electronic Schr\"odinger equation \eqref{eq:Hsolution1} yielding the effective potential for the nuclei can be solved. Finding an exact solution to this equation is, however, infeasible for more than a few electrons.

\subsubsection{Hohenberg-Kohn}


\subsection{Hellmann-Feynmann Theorem}
\subsection{Conclusion}

\section{Lattice Dynamics}
\subsection{Equations of Motion}
\subsection{Harmonic Approximation}
\subsubsection{Finite Differences}
\subsection{Harmonic Sampling}

\subsection{Molecular Dynamics}
\subsubsection{Thermodynamic Ensembles and Thermostats}
\subsubsection{Finite Temperature Equations of State and Lattice Expansion}
\subsubsection{Mode Projection}
\subsubsection{Approximative Anharmonic Methods}

\subsection{Heat Transport}
\subsubsection{Fluctuation Dissipation Theorem}
\subsubsection{Green and Kubo}
\subsubsection{Ab initio Virial Heat Flux}
\subsubsection{Ab initio Green Kubo}

\chapter{Screening Materials for Anharmonicity}
\section{Anharmonicity Measure}
\section{Screening Material Space}
\subsubsection{Literature Review}
\subsection{Candidate Materials}

\chapter{Thermal Conductivities for Strongly Anharmonic Compounds}
\section{Overview: Results of Dataset}
\section{Discussion}
\subsubsection{Relation to Anharmonicity}
\subsubsection{Dynamical Effects}

\chapter{Conclusion}

\chapter{Outlook}

\bibliography{references}
\bibliographystyle{unsrt}

\end{document}
