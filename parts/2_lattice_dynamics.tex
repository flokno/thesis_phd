In the previous chapter, we have seen how the many-body problem can be decoupled into an electronic problem given by Eq.\,\eqref{eq:Hsolution1}, which can be solved in the framework of DFT, and a nuclear problem given by Eq.\,\eqref{eq:chi2} that describes the dynamical properties of the nuclei. This was achieved by means of the Born-Oppenheimer approximation where electron-nucleus interactions beyond a parametric dependence on each other is neglected~\cite{BornOppenheimer}.

We will now introduce two approximations to progress in the description of the nuclear dynamics. First, the \emph{harmonic approximation} in which the nuclear Schr\"odinger equation is solved for an approximated potential. Second, we treat the nuclei as particles, but on the full, non-truncated Born-Oppenheimer potential which we denote by $\mathcal V ({\bf R})$ in the following. This will lead to the formulation of \emph{ab initio molecular dynamics} (aiMD).

We begin by recalling the Schr\"odinger equation for the nuclear wavefunctions $\chi_s ({\bf R})$ initially defined in Eq.\,\eqref{eq:chi2},
\begin{align}
  \left( T^{\rm Nuc} + \mathcal V ({\bf R}) \right) \chi_s ({\bf R})
  = E_s \chi_s ({\bf R})~,
  \label{eq:BOSE}
\end{align}
where the index $s$ was introduced to label different eigenstates, and
\begin{align}
  T^{\rm Nuc}
    = \sum_I \frac{- \hbar^2}{2 M_I} \frac{\partial^2}{\partial {\bf R}_I^2}~,
  \label{eq:Tnuc2}
\end{align}
is the nuclear kinetic-energy operator as before.

\subsection{Harmonic Approximation}
The Born-Oppenheimer potential appearing in Eq.\,\eqref{eq:BOSE} is an ordinary function of the $3 N_{\rm Nuc}$ coordinates ${\bf R} = ({\bf R}_1, \ldots, {\bf R}_{N_{\rm Nuc}})$ and therefore can be expanded as a Taylor series about a given configuration ${\bf R}^0$,~i.\,e.,
\begin{align}
\begin{split}
  \mathcal V ({\bf R} = {\bf R}^0 + \Delta {\bf R})
    = \mathcal V ({\bf R}^0)
    &+ \sum_{I, \alpha} 
      \left. \frac{\partial V({\bf R})}{\partial R^\alpha_I} 
      \right\vert_{{\bf R}^0}
    \,\Delta R^\alpha_I
    \\
    &
    + \frac{1}{2}
    \sum_{\substack{I, J \\ \alpha, \beta}}
    \left.\frac{\partial^{2} \mathcal{V}(\mathbf{R})}{\partial R_{I}^{\alpha} \partial R_{J}^{\beta}}\right|_{\mathbf{R}^{0}}
    \, \Delta R_I^\alpha \Delta R_J^\beta
    \\
    &+\frac{1}{3!}\cdots ~,
\end{split}
\end{align}
where the expansion coefficients are called \emph{force constants}. In particular, we have
\begin{align}
  \Phi_{\alpha, \beta}^{I, J}
    =\left.\frac{\partial^{2} \mathcal{V}(\mathbf{R})}{\partial R_{I}^{\alpha} \partial R_{J}^{\beta}}\right|_{\mathbf{R}^{0}}~,
\end{align}
the \emph{harmonic force constants}.

\REM{improve, Maradudin?}

\newpage

\subsection{Harmonic Approximation}
\subsubsection{Finite Differences}
\subsection{Harmonic Sampling}

\subsection{Molecular Dynamics}
We proceed by considering a semiclassical approximation to the nuclear wavefunction $\chi_s$ by writing~\CITE{Dirac, Landau, Marx}
\begin{align}
	\chi_s({\bf R}, t) = A({\bf R}, t) \, {\rm e}^{\frac{\im}{\hbar} S({\bf R}, t)}~,
	\label{eq:class.1}
\end{align}
where 

\CITE{Basdevant: Ehrenfest}

\subsubsection{Thermodynamic Ensembles and Thermostats}
\subsubsection{Finite Temperature Equations of State and Lattice Expansion}
\subsubsection{Mode Projection}
\subsubsection{Approximative Anharmonic Methods}

\subsection{Heat Transport}
\subsubsection{Fluctuation Dissipation Theorem}
\subsubsection{Green and Kubo}
\subsubsection{Ab initio Virial Heat Flux}
\subsubsection{Ab initio Green Kubo}