In the previous chapter, we have seen how the many-body problem can be decoupled into an electronic problem given by Eq.\,\eqref{eq:Hsolution1}, which can be solved in the framework of DFT, and a nuclear problem given by Eq.\,\eqref{eq:chi2} that describes the dynamical properties of the nuclei. This was achieved by means of the Born-Oppenheimer approximation where electron-nucleus interactions beyond a parametric dependence on each other is neglected~\cite{BornOppenheimer}.

\subsection{Equations of Motion}
\subsection{Harmonic Approximation}
\subsubsection{Finite Differences}
\subsection{Harmonic Sampling}

\subsection{Molecular Dynamics}
\subsubsection{Thermodynamic Ensembles and Thermostats}
\subsubsection{Finite Temperature Equations of State and Lattice Expansion}
\subsubsection{Mode Projection}
\subsubsection{Approximative Anharmonic Methods}

\subsection{Heat Transport}
\subsubsection{Fluctuation Dissipation Theorem}
\subsubsection{Green and Kubo}
\subsubsection{Ab initio Virial Heat Flux}
\subsubsection{Ab initio Green Kubo}