In the previous chapter, we have seen how the many-body problem can be decoupled into an electronic problem given by Eq.\,\eqref{eq:Hsolution1}, which can be solved in the framework of DFT, and a nuclear problem given by Eq.\,\eqref{eq:chi2} that describes the dynamical properties of the nuclei. This was achieved by means of the Born-Oppenheimer approximation where electron-nucleus interactions beyond a parametric dependence on each other is neglected~\cite{BornOppenheimer}.

We will now introduce two approximations to progress in the description of the nuclear dynamics. First, the \emph{harmonic approximation} in which the nuclear Schr\"odinger equation is solved for an approximated potential. Second, we treat the nuclei as particles, but on the full, non-truncated Born-Oppenheimer potential which we denote simply by $\mathcal V ({\bf R})$ in the following. This will lead to the formulation of \emph{ab initio molecular dynamics} (aiMD).

We begin by recalling the Schr\"odinger equation for the nuclear wavefunctions $\chi_s ({\bf R})$ initially defined in Eq.\,\eqref{eq:chi2},
\begin{align}
  \left( T^{\rm Nuc} + \mathcal V ({\bf R}) \right) \chi_s ({\bf R})
  = E_s \chi_s ({\bf R})~,
  \label{eq:BOSE}
\end{align}
where the index $s$ is introduced to label different eigenstates, and
\begin{align}
  T^{\rm Nuc}
    = \sum_I \frac{- \hbar^2}{2 M_I} \frac{\partial^2}{\partial {\bf R}_I^2}~,
  \label{eq:Tnuc2}
\end{align}
is the nuclear kinetic-energy operator as before.

\subsection{Harmonic Approximation}
The Born-Oppenheimer potential $\mathcal V ({\bf R})$ appearing in Eq.\,\eqref{eq:BOSE} is an ordinary function of the $3 N_{\rm Nuc}$ coordinates ${\bf R} = ({\bf R}_1, \ldots, {\bf R}_{N_{\rm Nuc}})$ and therefore can be expanded as a Taylor series about a given configuration ${\bf R}^0$,~i.\,e.,
\begin{align}
\begin{split}
  \mathcal V ({\bf R} = {\bf R}^0 + \Delta {\bf R})
    = \mathcal V ({\bf R}^0)
    &+ \sum_{I, \alpha} 
      \left. \frac{\partial V({\bf R})}{\partial R^\alpha_I} 
      \right\vert_{{\bf R}^0}
    \,\Delta R^\alpha_I
    \\
    &
    + \frac{1}{2}
    \sum_{\substack{I, J \\ \alpha, \beta}}
    \left.\frac{\partial^{2} \mathcal{V}(\mathbf{R})}{\partial R_{I}^{\alpha} \partial R_{J}^{\beta}}\right|_{\mathbf{R}^{0}}
    \, \Delta R_I^\alpha \Delta R_J^\beta
    \\
    &+\frac{1}{3!}\cdots ~,
\end{split}
\end{align}
where the expansion coefficients are called \emph{force constants}. In particular, we have
\begin{align}
  \Phi_{\alpha, \beta}^{I, J}
  \equiv \left.\frac{\partial^{2} \mathcal{V}(\mathbf{R})}{\partial R_{I}^{\alpha} \partial R_{J}^{\beta}}\right|_{\mathbf{R}^{0}}~,
\end{align}
i.\,e.,~the \emph{harmonic force constants}. The harmonic approximation is typically used to assess dynamical properties of a system in some confined phase-space region close a (local) minimum of the potential-energy surface. It is therefore customary to start the investigation from the ground state of the system,~i.\,e.,~a local minimum configuration ${\bf R}^0$ characterized by
\begin{align}
	\left. \frac{\partial V({\bf R})}{\partial R^\alpha_I} 
	\right\vert_{{\bf R}^0} 
		&~=~ 0 \quad\text{for all}\quad (I, \alpha),~\text{and} \\
	\sum_{\substack{I, J \\ \alpha, \beta}}
	\Phi_{\alpha, \beta}^{I, J}
	\, \Delta R_I^\alpha \Delta R_J^\beta
		&~>~ 0 \quad\text{for all possible}\quad \set{\Delta {\bf R}_I}~.
	\label{eq:ha.positive}
\end{align}
The condition in Eq.\,\eqref{eq:ha.positive} is fulfilled if the harmonic force constants $\Phi_{\alpha, \beta}^{I, J}$ are positive-definite.

We progress by defining \emph{mass-reduced coordinates} for the displacements
\begin{align}
	{\bf q}_I 
		&\equiv \sqrt{M_I} \Delta {\bf R}_I~, 
		\label{eq:uI} \\
	{\bf p}_I 
		&\equiv \frac{-\im \hbar}{\sqrt{M_I}} \frac{\partial}{\partial {\bf q}_I}~,
		\label{eq:pI} \\
	D_{\alpha, \beta}^{I, J}
		&\equiv \frac{1}{M_I M_J} \Phi_{\alpha, \beta}^{I, J}~,
		\label{eq:D}
\end{align}
where Eq.\,\eqref{eq:D} defined the \emph{dynamical matrix }$\rm D$.
Using these coordinates, the Hamiltonian assumes the form
\begin{align}
	\begin{split}
		\mathcal{H}({\bf q}, {\bf p})
			&= T^{\rm Nuc} ({\bf p}) + \mathcal{V}^{(2)} ({\bf q})\\
			&= \halb \sum_I {\bf p}_I^2 + 
				\halb \sum_{\substack{I, J \\ \alpha, \beta}}
					D_{\alpha, \beta}^{I, J}
					\, q_I^\alpha q_J^\beta~.
	\end{split}
	\label{eq:ha.H1}
\end{align}
As required earlier, the dynamical matrix $\rm D$ is a positive definite \REM{tbc}


\subsubsection{Finite Differences}
\subsection{Harmonic Sampling}

\subsection{Molecular Dynamics}
The classical limit of the nuclear Schr\"odinger equation~\eqref{eq:BOSE} is usually performed by writing the nuclear wavefunction $\chi_s ({\bf R}, t)$ in terms of a real amplitude $A_s({\bf R}, t)$ and a \emph{classical action function} $S_s({\bf R}, t)$~\cite{Dirac1981,Landau2013,Marx2009}
\begin{align}
	\chi_s({\bf R}, t) = A_s({\bf R}, t) \, {\rm e}^{\frac{\im}{\hbar} S_s({\bf R}, t)}~.
	\label{eq:class.1}
\end{align}
The Schr\"odinger equation then yields a set of differential equations for $A_s$ and $S_s$ that, in the limit $\hbar \to 0$, go over to a \emph{Hamilton-Jacobi} equation for $S_s$,~i.\,e.,~
\begin{align}
  \frac{\partial S_s}{\partial t} + \mathcal H \left({\bf R}, {\bf P}\right)
  = 0~,
  \label{eq:HamiltonJacobi}
\end{align}
where ${\bf P} = ({\bf P}_1, \ldots) \equiv ({\bf \nabla}_1 S_s, \ldots)$ denotes the conjugate momenta and $\mathcal H$ is the \emph{classical} Hamilton function corresponding the to the operator in Eq.\,\eqref{eq:BOSE}, from which the equations of motion for the nuclei can be obtained:
\begin{align}
  \dot{{\bf P}}_I 
    = -\frac{\partial \mathcal H}{\partial {\bf R}_I}
    \quad\implies\quad M_I \ddot{\bf R}_I
    = -\frac{\partial \mathcal V}{\partial {\bf R}_I}~.
    % \equiv - {\bf F}_I~.
\end{align}
The negative gradient of the Born-Oppenheimer potential, 
$-\partial \mathcal V / \partial {\bf R}_I$ is the force ${\bf F}_I$ acting on atom $I$ which can be obtained via the Hellmann-Feynmann theorem,~cf.~Sec.\,\ref{sec:HellmannFeynman}.

An alternative viewpoint that is more instructive can be taken by invoking the \emph{Ehrenfest theorem}~\cite{Ehrenfest1927,Basdevant2007}. The statement is that
\begin{align}
  \frac{\d}{\d t} \left\langle {\bf P}_I \right\rangle_{\chi_s}
    = \left\langle
      - \frac{\partial \mathcal{V}}{\partial {\bf R}_I}
    \right\rangle_{\chi_s}~,
  \label{eq:ehrenfest.de1}
\end{align}
where $\langle \cdot \rangle_{\chi_s}$ denotes an expectation value taken with respect to the state $\chi_s$. This expression differs only slightly from the classical counterpart, which would read
\begin{align}
\frac{\d}{\d t} \left\langle {\bf P}_I \right\rangle
= \left.
- \frac{\partial \mathcal{V}}{\partial {\bf R}_I}
\right\vert_{{\bf R} = \langle {\bf R} \rangle}~.
\label{eq:ehrenfest.de2}
\end{align}
The difference comes from the fact that, in general,
\begin{align}
  \delta f
  \equiv 
  f \bm ( \langle x \rangle \bm{)} 
  - 
  \bm{\langle} f (x) \bm{\rangle}
  \neq 0
  ~,
  \label{eq:ehrenfest.delta1}
\end{align}
where $x = {\bf R}_I$ denotes the space coordinate for notional simplicity, $f$ is some function of the observable $x$, and $\delta f$ measures the difference between the classical and the quantum expectation value. Ehrenfest's argument is that this difference becomes negligible when the state is sufficiently peaked around some value $x_0$. Expanding $f$ around the expectation value of $x$, $x_0 \equiv \langle x \rangle$, we have
\begin{align}
  f(x) = f \bm ( \langle x \rangle \bm{)}  
    + (x - \langle x \rangle) \, f' \bm ( \langle x \rangle \bm{)}
    + \frac{1}{2} (x - \langle x \rangle)^2 \, f'' \bm ( \langle x \rangle \bm{)}
    + \cdots~.
  \label{eq:ehrenfest.f2}
\end{align}
It follows that the $f'$ term vanishes when the expectation value is taken, and
\begin{align}
\langle f(x) \rangle 
  = f \bm ( \langle x \rangle \bm{)}  
    + \frac{1}{2} \Delta x^2 f'' \bm ( \langle x \rangle \bm{)}
    + \cdots~,
\label{eq:ehrenfest.f3}
\end{align}
where $\Delta x^2 = \bm{\langle} (x - \langle x \rangle)^2 \bm{\rangle}$ measures the variance of the underlying distribution,~i.\,e.~the width of the wavepacket. The relative error between the classical and quantum expectation value is readily computed to be
\begin{align}
  \left\lvert \frac{\delta f}{f \bm ( \langle x \rangle \bm{)}} \right\rvert
  = \frac{1}{2} \Delta x^2 \left\lvert \frac{f'' \bm ( \langle x \rangle \bm{)}}{f \bm ( \langle x \rangle \bm{)}} \right\rvert
+ \mathcal{O}(\Delta x^3)~.
  \label{eq:ehrenfest.delta2}
\end{align}
This estimation holds in general for any observable $f$.
By crudely estimating the dimension of the wavepacket in terms of the thermal de Broglie-wavelength, we find
\begin{align}
  \Delta x^2 
    \sim \left( \frac{h}{P} \right)^2
    \sim \frac{h^2}{MT}~,
  \label{eq:ehrenfest:dimension}
\end{align}
which gives support to the intuitive assumption that we can expect the classical limit to work better the heavier the atoms and the higher the temperature.
Let us now set $f(x) \hat = -\partial \mathcal V / \partial {\bf R}_I$, then another important conclusion can be drawn from Eq.\,\eqref{eq:ehrenfest.delta2}: For a harmonic potential $\mathcal V ({\bf R}) = \mathcal V^{(2)} ({\bf R})$, where derivatives higher than second order vanish, the classical and quantum mechanical expectation values \emph{always coincide}. The quantum mechanical expectation value of position will therefore evolve in the same time-periodic fashion as a classical particle in a harmonic well.

\REM{this approach can only be validated by computing observables and compare the results. One can safely say that this approach has been used successfully in a plethora of studies, while additional care must be taken at low temperature and/or systems with light atoms, especially hydrogen-bonded systems~\CITE{MD-Review?,MarklandCeriotti,Litman,pHexperiment}.}

\subsubsection{Thermodynamic Ensembles and Thermostats}
\subsubsection{Finite Temperature Equations of State and Lattice Expansion}
\subsubsection{Mode Projection}
\subsubsection{Approximative Anharmonic Methods}

\subsection{Heat Transport}
\subsubsection{Fluctuation Dissipation Theorem}
\subsubsection{Green and Kubo}
\subsubsection{Ab initio Virial Heat Flux}
\subsubsection{Ab initio Green Kubo}