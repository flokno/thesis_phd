\chapter{Danksagungen / Acknowledgements}

Ich möchte zunächst den wichtigsten Lehrern im Rahmen meiner nun mehrjährigen Beschäftigung mit Physik danken: Peter Richter, Bálint Aradi, und Paul Gartner, die mein Denken über Physik entscheidend geprägt haben und die für mich als Wissenschaftler und Personen große Vorbilder sind.

Ein ebenso großer Dank gilt Matthias Scheffler, der mir die Möglichkeit eröffnet hat über mehrere Jahre Teil des FHIs zu sein und so einen völlig neuen Einblick in die Welt der Wissenschaft zu erlangen, den ich so aus meinem Studium nicht kannte. Diese Arbeit und alles was daraus folgt wäre ohne seinen Anstoß niemals entstanden.

Ebenfalls möchte ich Christian Carbogno danken, der mir mit einer Mischung aus Freie-Hand-Lassen und konkretem Feedback geholfen hat sicher durch die Zeit am FHI zu manövrieren.

Weiterer Dank gilt meinen Mitstreiter:innen am FHI: Yair Litman, von dem ich sehr viel gelernt habe und der meinen Respekt vor Chemie empfindlich gesteigert hat -- es war mir eine große Ehre. Sebastian Kokott für viele, machmal hitzige Diskussionen rund um DFT, und nicht zuletzt Marcel Langer für die spannende Zusammenarbeit mit gknet und vibes, aber auch für viele ``kleine Biere''. Julia Pach, Hanna Krauter, und Steffen Kangowski danke ich für einen vorbildlichen Organisationsapparat den man sich kaum besser wünschen kann. 
Furthermore, I want to thank Thomas Purcell for all the hard work put into FHI-vibes and related projects, and Dima and his friends for a great deal of fun. You all will be remembered.

Ich möchte Andreas Prokop dafür danken, dass er früh mein Interesse für Mathematik und Naturwissenschaften geweckt und meinen Werdegang stets interessiert verfolgt hat, und in diesem Zuge wäre auch Frank Brand zu nennen, der mich überhaupt erst auf die Idee brachte Physik zu studieren.

Before concluding, I want to thank Olle Hellman for showning me a way to move on.

Schließlich möchte ich meinen Eltern für ihre bedingungslose Unterstützung in allem was ich tue danken.