\epigraph{\singlespacing \it ``Die Zeit des unbedenklichen Wirtschaftens mit den Energiequellen und Stofflagern, die uns die Natur zur Verfügung gestellt hat, wird wahrscheinlich schon für unsere Kinder nur noch die Bedeutung einer vergangenen Wirtschaftsepoche haben.''}{W. Schottky, 1929~\cite{Schottky1929}}
% [ ] Why thermal conductivity?
% [ ] Why novel thermal insulators?
% [ ] Why computational materials science?
\newthought{One of the major challenges} humankind faces in the 21th century is the responsible and sustainable handling of the earth's natural resources.  Yet, most energy today is lost as waste heat during the transformation of raw energy sources to usable power. To date, there is no fuel based heat engine that exceeds an efficiency of 50\,\% and often it is even worse~\cite{eia}. 
Since gas- and aircraft-turbines are essentially Carnot engines, their efficiency and core power are directly related to combustion temperature~\cite{Clarke2012,Perepezko2009}. This relationship has been taken advantage off during the past 30 years by developing 
ceramics with high thermal resistivity that are nowadays applied as \emph{thermal barrier coatings} on turbine airfoils in heat engines: thin heat insulating layers that allow to operate a turbine at higher temperatures, thereby increasing its efficiency~\cite{Clarke2003}.

A complementary strategy is to recycle waste heat where it occurs. One way of achieving this is by using the \emph{thermoelectric effect} to  generate electric power from temperature gradients~\cite{Snyder2008}. The main obstacle preventing mass operation is the limited conversion rate (figure of merit) $zT$ of even the most advanced thermoelectric materials known to date. To make matters worse, these materials often contain heavy metals and are toxic, and their manufacturing process is difficult and expensive~\cite{Nolas2001}. Recent advancements in the field, such as the discovery of a high thermoelectric figure of merit in the lead-free material Tin Selenide~\cite{zhao2014}, offer hope that novel materials with significant figure of merit that are non-toxic, easy and cheap to produce, and consist of abundant elements, can be found.

\newthought{A key physical property} of both thermal barrier coatings (TBCs) and thermoelectrics is their thermal conductivity $\kappa$. In the case of thermoelectrics, the figure of merit is inversely proportional to $\kappa$~\cite{Nolas2001}:
\begin{align}
zT = \frac{S^2 \sigma_{\rm el}}{\kappa} T~,
\label{eq:zT}
\end{align} 
where $T$ dentoes the temperature, $S$ the Seebeck coefficient, and $\sigma_{\rm el}$ is the electrical conductivity.
A prerequisite to finding better thermoelectrics or TBCs therefore is to find materials which are thermally insulating. These are typically non-metals, since the free electrons in metals are good heat carriers, and most of the known thermoelectrics are thermally insulating inorganic semiconductors~\cite[p.\,15]{Nolas2001}.

Despite the technological need, systematic knowledge of thermal conductivities in inorganic compounds is scarce. A database like Springer Materials only lists thermal conductivities for about 200~of these compounds~\cite{SpringerMaterials}, which is partially due to the fact that accurate measurements of thermal conductivity are challenging to perform and reproducibility between different experimental groups is often not guaranteed~\cite{wei2016}. As a consequence, thermal conductivity is not systematically understood beyond semi-empirical and phenomenological trends in a very limited number of simple material classes~\cite{morelli2006}.

\newthought{The aim of this work} is therefore to open a new pathway for overcoming the problem of limited data by devising a route to systematically scan material space for thermal insulators and calculate their thermal conductivities from first principles. 

While the theoretical foundations of thermal transport in non-metals are about one hundred years old,\footnote{The many pitfalls in early attempts to describe thermal transport in semiconductors was summarized by Peierls in his memorial text in honor of Wolfgang Pauli~\cite{Peierls1960}.} the simulation of thermal conductivities with predictive accuracy from first principles only emerged in the past fifteen years~\cite{Broido2007}, and a fully non-perturbative treatment in terms of \emph{ab initio} Green-Kubo theory is available since five years~\cite{Marcolongo2016,Carbogno2016}. However, the number of solid materials computed by the latter approach is very small: Solid silicon and zirconia~\cite{Carbogno2016}, ice X~\cite{Grasselli2020}, and amorphous silica~\cite{Marcolongo2020}. As we will see later, thermal insulators are often strongly \emph{anharmonic} and require a non-perturbative treatment to describe their dynamical properties accurately. 

\newthought{The route of this work is therefore twofold:} After reviewing the relevant theoretical tools necessary to simulate heat transport in thermal insulators, we describe how to assess anharmonicity in a quantitative and paremeter-free way without the need for explicit model building beyond the harmonic approximation. In a second step, we use this ``measure of anharmonicity'' to identify candidate thermal insulators and compute non-perturbative thermal conductivities for them with nearly experimental accuracy.


\section*{Organization of the Thesis}
The thesis is split into two parts: The first part introduces the theoretical concepts necessary to understand thermal transport in materials from an \emph{ab initio} perspective.
In chapter one, we will introduce the quantum-mechanical many-body problem and describe the necessary steps and key approximations that lead to Kohn-Sham density functional theory as a way of solving the electronic problem in practice. Chapter two will describe the key concepts of nuclear dynamics that are necessary to study thermodynamical properties of materials, such as heat transport: The chapter introduces the harmonic approximation as a powerful starting point for studying dynamical propertis of matter, and the fully non-perturbative treatment of nuclear dynamics and thermodynamic properties in terms of molecular dynamics simulations. Chapter three is dedicated to heat transport theory in the framework of linear response as formulated in the Green-Kubo formalism. By the end of this chapter, it should be clear how heat transport emerges from the many-body Schr\"odinger equation, and how, in principle, thermal conductivity can be computed from first principles.

The second part can be understood as an application of this body of theory and is devoted to the investigation of materials in the context of heat transport. In chapter four, we introduce a novel concept to quantify the anharmonicity of material as a means to detect materials that should be treated non-perturbatively. As we will see, this quantitity directly correlates with a material's thermal conductivity and therefore enables to predict candidate thermal insulators. Chapter five is devoted to introducing the technical details necessary to run \emph{ab initio} Green Kubo (aiGK) simulations in practice. In chapter six, we present results for \CHECK 60~materials, first for \CHECK 20 where experimental reference is available to benchmark the aiGK method, then for the remaining materials where thermal conductivity was previously unknown.

After discussing our results, we conclude with a summary and an outlook on new questions that arose in the course of this work.
