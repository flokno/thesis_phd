\chapter{Abstract}

\section*{English}

Heat transport is an important phenomenon in many branches of physics and adjacent fields, be it astrophysics and earth sciences, where thermodynamic properties of planets are studied, or materials science investigating technologically relevant compounds. In dielectric solids, the most important contribution to heat transport comes from the transfer of vibrational energy of atoms -- heat -- mediated by the interatomic bonding. The simplest model to describe this bonding is the harmonic approximation,~i.\,e.,~the description of atom bonds as perfect springs. However, the harmonic approximation is incapable of describing thermal conductivity in periodic systems: A perfectly harmonic, defect-free crystal would approach vanishing thermal resistance in the bulk limit. Finite thermal conductivity in realistic systems is a consequence of deviations from the harmonic description of atom bonds: Anharmonicity. Depending on the strength of anharmonic contributions to the interatomic bonding, these can be captured as a small correction to the harmonic approximation in the framework of perturbation theory, or require a non-perturbative description once they become too strong.

In this work, we describe how a non-perturbative heat transport formalism for solids emerges in the framework of \emph{ab initio} simulations coupled with linear response theory. The resulting \emph{ab initio} Green Kubo method allows for studying heat transport in solids of arbitrary anharmonic strength, and is particularly suited to describe ``strongly anharmonic'' systems where perturbative approaches become unreliable. In order to discern harmonic from anharmonic materials in a systematic way, we introduce an ``anharmonicity measure'' which  quantifies the anharmonic contribution to the interatomic forces under thermodynamic conditions. Using this anharmonicity measure, we investigate typical dynamical effects occurring in strongly anharmonic compounds and investigate the limits of perturbative approaches for the study of thermal transport. We show that this measure negatively correlates with bulk thermal conductivities in simple solids, supporting the intuitive notion that more harmonic materials are better heat conductors and vice versa. Based on these findings, we identify anharmonic compounds as candidates for thermal transport simulations in the search for novel thermal insulators. In this way, we identify several new thermal insulators of potential technological relevance as thermal barriers or thermoelectric materials which we suggest for experimental study.

\section*{Deutsch}

Wärmetransport ist ein wichtiges Phänomen in vielen Bereichen der Physik und angrenzender Gebiete, sein es Astrophysik und Geowissenschaften, die thermodynamische Eigenschaften von Planeten untersuchen, oder Materialwissenschaften, die sich mit technologisch relevanten Stoffen auseinandersetzen. In dielektrischen Festkörpern stammt der wichtigste Beitrag zu Wärmetransport vom Transfer der Vibrationsenergie der Atome -- Wärme -- vermittelt durch die interatomaren Bindungen. Das einfachste Modell um diese Bindungen zu beschreiben ist die harmonische Näherung,~d.\,h.,~die Beschreibung von Atombindungen als perfekte Federn. Die harmonische Näherung ist jedoch nicht geeignet um Wärmeleitfähigkeit in periodischen Systemen zu beschreiben: In perfekt harmonischen, defektfreien Kristallen würde der thermische Widerstand im thermodynamischen Limes verschwinden. Eine endliche thermische Leitfähigkeit in realistischen Systemen ist die Konsequenz von Abweichungen von der harmonischen Beschreibung der Atombindungen: Anharmonizität. Abhängig von der Stärke des anharmonischen Beitrags zur Atombindung kann diese als kleine Korrektur zur harmonischen Näherung im Rahmen von Störungstheorie beschrieben werden, oder erfordert eine nicht-störungstheoretische Behandlung falls sie zu stark wird.

In dieser Arbeit beschreiben wir wie nicht-störungstheoretischer Wärmetransport im Rahmen von \emph{ab initio}-Simulationen und linearer Antworttheorie formuliert werden kann. Die daraus resultierende \emph{ab initio}-Green-Kubo-Methode ermöglicht die Simulation von Wärmetransport in Festkörpern beliebiger Anharmonizität und ist besonders geeignet um ``stark anharmonische'' Systeme zu beschreiben in denen störungstheoretische Ansätze unzuverlässig werden. Um die systematische Unterscheidung von harmonischen und anharmonischen Materialien zu ermöglichen führen wir ein ``Anharmonizitätsmaß'' ein, welches die anharmonischen Beiträge zu den interatomaren Kräften unter thermodynamischen Bedingungen quantifiziert. Mit diesem Anharmonizitätsmaß untersuchen wir typische dynamische Effekte die in stark anharmonischen Materialien auftreten, sowie die Grenzen störungstheoretischer Methoden zur Berechnung von Wärmetransporteigenschaften. Wir zeigen, dass eine negative Korrelation des Anharmonizitätsmaßes mit der Wärmeleitfähigkeit einfacher Kristalle besteht, was die intuitive Auffassung bestärkt, wonach harmonische Materialien bessere Wärmeleiter sind und umgekehrt. Auf diesen Erkenntnissen aufbauend identifizieren wird anharmonische Materialien als Kandidaten für Wärmetransport-Simulationen auf der Suche nach neuen thermischen Isolatoren. Auf diesem Wege identifizieren wir mehrere neue thermische Isolatoren welche potentielle technologische Relevanz als thermische Barrieren oder Thermoelektrika aufweisen könnten, und schlagen diese zur experimentellen Untersuchung vor.