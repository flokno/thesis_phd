\section{Summary}

From a physical point of view, we have presented a systematic study of \emph{ab initio} thermal transport in experimentally known semiconductors and insulators, focusing on strongly anharmonic systems. To this end, we have developed a novel scheme based on first-principles force calculations which enables to measure the ``strength of anharmonicity'' in materials across chemical space, and facilitates to uncover strongly anharmonic dynamical effects in individual systems~\cite{Knoop2020}. We found that this measure of anharmonicty,~$\sigmaA$,~correlates significantly with experimental thermal conductivities, and inverted the logic to predict materials with potentially low thermal conducitivity based on estimating their anharmonic strength.

To study heat transport in these systems, we have presented a comprehensive exposition of classical Green-Kubo theory from first principles in the framework of DFT, and discussed the implementation of a slightly adapted version of the \emph{ab initio} Green Kubo (aiGK) method first presented by Carbogno and coworkers in Ref.~\cite{Carbogno2016} in FHI-vibes~\cite{FHI-vibes}. In Chp.~\ref{sec:results.experiments}, we have verified this approach by computing thermal conductivities at room temperature for 20~materials which are well characterized by experiments. We computed \todo{check} 37 more materials without experimental reference, finding \todo{check} 28 materials with low thermal conductivity $\kappa < 10$\,W/mK, with several materials in the range of state-of-the-art thermoelectrics $\leq 3$\,W/mK.

From a methodological point of view, we have presented a prototypical implementation of a data-driven approach for novel materials discovery: We leveraged existing knowledge to identify trends in material space based on an efficient descriptor, predicted candidate materials based on the descriptor, and studied the reduced number of materials by a high-accuracy method.

%\begin{itemize}
%	\item comprehensive exposition of (classical) GK theory from first principles in the framework of DFT
%	\item development of descriptor for identification of strongly anharmonic solids and effects~\cite{Knoop2020}
%	\item general implementation of aiGK method~\cite{Carbogno2016} in FHI-vibes~\cite{FHI-vibes}
%	\item aiGK results for XX materials
%		\subitem prediction of XX thermal conductivities in materials where no experiments were available before
%	\item prototypical implementation of a data-driven approach for novel material discovery
%		\subitem leverage experimental knowledge to identify trends in material space based on efficient descriptor
%		\subitem predict candidate materials based on descriptor
%		\subitem verify candidates by high-accuracy method
%\end{itemize}

\section{Outlook}

While the theoretical study of thermal transport in non-metals is about one hundred years old,\footnote{The many pitfalls in early attempts to describe thermal transport in semiconductors was summarized by Peierls in his memorial text in honor of Wolfgang Pauli~\cite{Peierls1960}.} the simulation of thermal conductivities with predictive accuracy from first principles only emerged in the past fifteen years~\cite{Broido2007}, and a fully non-perturbative treatment in terms of \emph{ab initio} Green-Kubo theory is available since about five years~\cite{Marcolongo2016,Carbogno2016}.

In light of the constant development of solid state physics, materials science, and computational sciences, we see two main routes for the further improvement of the status quo in \emph{ab initio} heat transport simulations from equilibrium approaches: Accelerated Green-Kubo simulations using machine-learned expressions for the potential energy function $\mathcal{V} ({\bf R})$~\cite{Korotaev2019,Li2020,Mangold2020}, and model Hamiltonian-based approaches using Boltzmann transport theory~\cite{Simoncelli2019} or analytical Green's functions~\cite{isaeva2019,dangic2021}.

\subsection{Machine learning Green Krubo}
The advent of machine-learned potentials with \emph{ab initio} quality~\cite{Behler2007,Bartok2010,Bartok2013,shapeev2016} coupled with on-the-fly or active-learning training strategies~\cite{Li2015,Jinnouchi2019,podryabinkin2017,Liu2021} promises to become a versatile tool for dynamical simulations of materials in a range of subfields, especially for phenomena where the electronic structure of a given material is only of secondary importance,~i.\,e.,~in uncharged systems with electronically trivial defects. In such materials, machine-learned potentials can remove computational bottlenecks when aiming for statistical convergence in simulation time, as well as system and ensemble sizes, which in turn could make GK-based transport simulations of new materials accessible to more researchers than currently.

While several proofs of principle for GK simulations from machine-learned potentials exist already~\cite{Korotaev2019,Li2020,Mangold2020}, some more fundamental problems persist,~e.\,g.,~the question how long-range electrostatic interactions in polar systems can be properly described in these models~\cite{Artrith2011,Grisafi2019,yue2021,kovacs2021}. From materials discovery perspective, where one aims at studying a multitude of systems, it remains to be seen how straightforward and robust the parametrization of these force fields can be achieved in practice, and if the extra cost in human time in order to train the potentials can be reduced to a minimum without compromising their reliability.
% There are also recent advances in incorporating nuclear quantum effects into GK approaches~\cite{luo2020,sutherland2021}

\subsection{Boltzmann transport and analytical Green's functions}
A complementary route for not too complex materials is the use of model Hamiltonians coupled with perturbation theory in the framework of lattice dynamics~\cite{Esfarjani2008,Hellman2013,Hellman2013b,errea2014,tadano2018,Zhou2019}. While these approaches are limited to non-diffusing materials with a definite long-range structural order and limited complexity,\footnote{Perturbation expressions scale \emph{at least} cubically with the size of the unit cell.} they allow to study subtle dynamical effects such as the the behavior of thermal transport close to and across phase transitions with very good precision~\cite{dangic2021}, which is particularly important for potential thermoelectrics such as SnSe and GeTe~\cite{Zhao2014,Dewandre2016,dangic2021}. Also, these methods give straightforward access to nuclear quantum effects~\cite{shulumba2017}, which is a formidable task for molecular dynamics-based Green Kubo methods~\cite{luo2020,sutherland2021}.

Worth noting are more recent approaches that try to bridge the gap between state of the art Boltzmann transport~\cite{Simoncelli2019} and analytical Green Kubo in terms of phonon Green's functions~\cite{isaeva2019,dangic2021}. Combined with self-consistent sampling techniques~\cite{brown2013}, these approaches promise to be a very efficient alternative for heat transport simulations in a wide class of systems.

Lastly, lattice-dynamics techniques can be coupled with molecular dynamics simulations to extract fully anharmonic properties such as phonon lifetimes,~e.\,g.,~as we use them in the \emph{ab initio} Green Kubo method~\cite{Ladd1986,Turney2009,Zhang2014,Carbogno2016,Glensk2019} -- the opportunities for combining these approaches are certainly not exhausted.

\subsection{Materials discovery}
Our screening for thermal insulators was governed by a single computational parameter,~i.\,e.,~the anharmonicity measure~$\sigmaA$. It is certain that including further structural and harmonic material properties in semi-empirical equations can improve thermal conductivity predictions and thereby accelerate materials discovery~\cite{ouyang2018,goldsmith2017,Chen2019,Purcell2021}.

%\subsection{Novel physics}
%At last, we would like to remark a few things about challenges from a physicist's point of view.
%
%\begin{itemize}
%	\item Methods:
%		\subitem ML potentials to make GK more feasible and accessible, remove computational bottleneck, go beyond GGA accuracy  
%		potentials:
%		\cite{Behler2007,Bartok2010,Bartok2013,shapeev2016}
%		long-range:
%		\cite{Artrith2011,Grisafi2019,yue2021,kovacs2021}
%		active learning/on the fly:
%		\cite{Li2015,Jinnouchi2019,podryabinkin2017,Liu2021}
%  	Shapeev: \cite{Korotaev2019},	DeepotSE: \cite{Li2020}, DonadioBehler: \cite{Mangold2020}
%		\subitem analytical GK: crossover between GK and BTE approach for not-too-complex materials
%		\cite{dangic2021,Simoncelli2019,isaeva2019}
%			\subsubitem NQE? \cite{shulumba2017,sutherland2021}
%		\subitem Question of reference positions, esp. close to phase transitions. Time-dependent?
%		\subitem 2D structures: People just do 3D transport, nobody really knows what's going on in 2D
%		\subitem Magnetism: No real theory for thermal transport in magnetic materials, only ad-hoc explanations like \cite{stockem2018}.
%		\subitem More generally: difficult systems for DFT, can be as banal as band-gap closing as observed for ZnSb
%	\item materials discovery
%		\subitem highlight that $\sigmaA$ is able to describe main trend in material space, improving on that by including more descriptors very well possible, will enable to accelerate novel materials discovery for thermal insulators
%			\CITE{Purcell?,Ramprasad,SISSO,Subgroup discovery,...}
%			\cite{ouyang2018,goldsmith2017,Chen2019}
%		
%		\subitem Screening depends crucially on the choice of the screened materials,~i.\,e.,~materials not included in the screening cannot be found.
%		
%		\subitem only bulk systems considered \cite{ferrando2020}
%\end{itemize}