\section{Summary}

In summary, we have studied \emph{ab initio} thermal transport.

\begin{itemize}
	\item comprehensive exposition of (classical) GK theory from first principles in the framework of DFT
	\item development of descriptor for identification of strongly anharmonic solids and effects~\cite{Knoop2020}
	\item general implementation of aiGK method~\cite{Carbogno2016} in FHI-vibes~\cite{FHI-vibes}
	\item aiGK results for XX materials
		\subitem prediction of XX thermal conductivities in materials where no experiments were available before
	\item prototypical implementation of a data-driven approach for novel material discovery
		\subitem leverage experimental knowledge to identify trends in material space based on efficient descriptor
		\subitem predict candidate materials based on descriptor
		\subitem verify candidates by high-accuracy method
\end{itemize}

\section{Outlook}

There are many things left to do.

\begin{itemize}
	\item Methods:
		\subitem ML potentials to make GK more feasible and accessible, remove computational bottleneck, go beyond GGA accuracy  
			\CITE{Carla,DonadioBehler,DeepotSE,Langer,...}
		\subitem analytical GK: crossover between GK and BTE approach for not-too-complex materials
			\CITE{Dangic,Simoncelli}
			\subsubitem NQE? \cite{shulumba2017,sutherland2021}
	\item materials discovery
		\subitem highlight that $\sigmaA$ is able to describe main trend in material space, improving on that by including more descriptors very well possible, will enable to accelerate novel materials discovery for thermal insulators
			\CITE{Purcell?,Ramprasad,SISSO,Subgroup discovery,...}
\end{itemize}