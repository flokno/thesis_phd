\section*{Summary}

We have presented a systematic study of \emph{ab initio} thermal transport in experimentally known semiconductors and insulators, focusing on strongly anharmonic systems. To this end, we have developed a novel scheme based on first-principles force calculations which enables to measure the ``strength of anharmonicity'' in materials across chemical space, and facilitates to uncover strongly anharmonic dynamical effects in individual systems in a computationally efficient way~\cite{Knoop.2020}.
\mscomment{tell why this is better than Grueneisen or width of peaks in g(r)}
 We found that this measure of anharmonicty,~$\sigmaA$,~correlates significantly with experimental thermal conductivities, and used the logic to predict materials with potentially low thermal conducitivity based on estimating their anharmonic strength.

To study heat transport in these systems, we have presented a comprehensive exposition of classical Green-Kubo theory from first principles in the framework of DFT, and discussed the implementation of a slightly adapted version of the \emph{ab initio} Green Kubo (aiGK) method first presented by Carbogno, Ramprasad and Scheffler in Ref.~\cite{Carbogno.2016} in FHI-vibes~\cite{FHI-vibes}. In Chp.~\ref{sec:results.experiments}, we have verified this approach by computing thermal conductivities at room temperature for 24~materials which are well characterized by experiments. 
\mscomment{add careful comparison to BT for MgO and Si}
\FK{MgO is done, I can improve a little bit. We didn't study Si this time.}
We computed 33 more materials without experimental reference, finding 28 materials with low thermal conductivity $\kappa < 10$\,W/mK, with several materials in the range of state-of-the-art thermoelectrics $\leq 2$\,W/mK, in particular the class of chalcopyrite materials discussed in Sec.\,\ref{sec:chalcopyrites}.

The number of materials studied in this work with fully non-perturbative \emph{ab initio} Green Kubo theory of thermal transport is therefore an order of magnitude higher than all previously published results for solid systems combined. These comprise solid silicon and zirconia~\cite{Carbogno.2016}, ice X~\cite{Grasselli.2020}, and amorphous silica~\cite{Marcolongo.2020}.\footnote{Further aiGK calculations have been published for liquids: Liquid Argon, heavy water, and water in different phases~\cite{Marcolongo.2016,Marcolongo.2020,Grasselli.2020}.}
\mscomment{Was this discussed earlier?}
\FK{Yes in the introduction.}

\newthought{From a methodological point of view}, we have presented a prototypical implementation of a data-driven approach for novel materials discovery: We leveraged existing knowledge to identify trends in material space based on an efficient descriptor, predicted candidate materials based on the descriptor, and studied the reduced number of materials by a high-accuracy method.
\mscomment{This was done >40 yrs ago}
\FK{I didn't say I invented the approach. Delete?}

%\begin{itemize}
%	\item comprehensive exposition of (classical) GK theory from first principles in the framework of DFT
%	\item development of descriptor for identification of strongly anharmonic solids and effects~\cite{Knoop.2020}
%	\item general implementation of aiGK method~\cite{Carbogno.2016} in FHI-vibes~\cite{FHI-vibes}
%	\item aiGK results for XX materials
%		\subitem prediction of XX thermal conductivities in materials where no experiments were available before
%	\item prototypical implementation of a data-driven approach for novel material discovery
%		\subitem leverage experimental knowledge to identify trends in material space based on efficient descriptor
%		\subitem predict candidate materials based on descriptor
%		\subitem verify candidates by high-accuracy method
%\end{itemize}

\section*{Outlook}
To conclude the thesis, we want to give a short outlook on topics and questions that naturally arise from this project.


\subsection*{Remaining open questions}
\label{sec:outlook.open_questions}
One question that is not fully answered after this project is at which degree of anharmonicity fully non-perturbative Green Kubo calculations become necessary to compute accurate thermal conductivities, and when perturbative treatment in terms of cubic or cubic and quartic anharmonic contributions in the framework of self-consistent or effective phonons is sufficient~\cite{Hellman.2013b,Feng.2016,Tadano.2018zm7,Xia.2018,Ravichandran.2018}. The conceptual tools for studying this question have been laid out completely in this work: The anharmonicity measure~$\sigmaA$ can be generalized to quantify third, fourth, and higher order anharmonicity separately in straightforward manner. Furthermore, the molecular dynamics data produced in this work are made accessible to the community so that force constants models as input for perturbative expressions of thermal conductivity can be extracted with regression or sensing approaches~\cite{Zhou.2014,Fransson.2020}. And of course, the thermal conductivities computed in this work can serve as benchmark for identifying materials with significant deviations which are suited for further testing. This should be possible as of now at least for the binary systems with high symmetry studied in this work, as for more complex systems a treatment of quartic anharmonicity might become infeasible because of the unfavorite scaling of quartic force constants with number of irreducible atoms, see the discussion in appendix D of Ref.~\cite{Ravichandran.2018}.
\mscomment{This was the main motivation for this work.}
\FK{I would say: when ever you can afford it, use self-consistent or effective phonons. The actual problem from my current perspective is that perturbation theory scales far worse than GK. There is a reason that Xia et al. publish on the order of 40 simple cubic systems when using modern BTE approaches.}

\subsection*{Next steps for materials discovery}
Our screening for thermal insulators was governed by a single computational parameter,~i.\,e.,~the anharmonicity measure~$\sigmaA$. It is certain that including further structural and harmonic material properties in semi-empirical equations derived by feature extraction techniques can improve thermal conductivity predictions and thereby accelerate materials discovery~\cite{Ouyang.2018,Goldsmith.2017,Chen.2019,Purcell2021}. The systematic data computed in the course of this project can serve as a testbed for these approaches.

\subsection*{Next steps for ab initio Green Kubo}
As presented in the previous chapters, the \emph{ab initio} Green Kubo method in its current formulation uses an interpolation approach to deal with long-wavelength phonons, assuming an approximative scaling $\tau_{\b q, b} \propto \omega^{-2}_{\b q, b}$, where $\tau_{\b q, b}$ denotes the lifetime and $\omega_{\b q, b}$ the angular frequency of a phonon with wave vector $\b q$ and band index $b$. This relationship, however, only holds for crystals of certain lattice types in the limit of vanishing wave vector and small anharmonicity~\cite{Herring.1954}. Especially when working on intermediate levels of anharmonicity, where phonons with long wavelengths are more important, an improvement of the current interpolation scheme is certainly desirable,~e.\,g.,~by using the full phonon spectral function $S(\b q, \omega)$ which implicitly contains information about frequencies $\omega_{\b q, b}$ and lifetimes $\tau_{\b q, b}$, lends itself for interpolation in momentum space, and can be systematically improved by surrogate models beyond the harmonic approximation~\cite{Maradudin.1962}. More advanced surrogate models could also help to map out long-lived contributions better, thereby reducing the necessary amount of effective simulation time, $\teff$, by converging out the harmonic contribution to heat transport faster than in the current approach. The data produced in the course of this work can serve as a basis to develop and test ideas in that regard.


\subsection*{Challenges for ab initio Green Kubo}
We see the following challenges with the current formulation of the \emph{ab initio} Green Kubo method that are worth further investigation: 

\newthought{The issues of defects and isotope scattering} have only been briefly mentioned in the discussion of thermal transport in MgO in Sec.~\ref{sec:mgo.experiments}, but have not been further investigated in this work, although they are known to impact thermal conductivity in actual materials~\cite{Bisson.2000}. In a supercell-based \emph{ab initio} approach, these effects are notoriously difficult to study because of the required system sizes and time scales~\cite{Gibbons.2011}. However, these are technical and not conceptual issues which might be possible to solve by the increasing computational power, or by surrogate models based on a DFT description of the potential-energy surface, see also Sec.\,\ref{sec:outlook.ml}.
\mscomment{your cells are large enough to study these effects}
\FK{isotope scattering for MgO: probably yes. Technologically relevant defect concentrations: probably no.}

\newthought{The topic of convective contributions to the heat flux} has been touched in Chp.~\ref{chp:heat_transport}. As these are absent in the virial-based heat flux formulation used in this work, this rules out any study of materials with noticeable self diffusion. Furthermore, as discussed in Sec.~2.3.1 and appendix A of Ref.~\cite{ErcoleThesis}, there is no rigorous mathematical proof for the assumption of vanishing convective contributions to the thermal conductivity even in system without any self diffusion, since those can, in principle, contribute through the cross-correlation of convective and non-convective currents. While those are often negligible~\cite{Vogelsang.1987}, it would be interesting to estimate the strength of this effect. One could compute convective contributions to the heat flux based on a force constants model, and evaluate its contribution. This should be sufficient to quantify the expected deviation, and materials with noticeable deviation could be interesting to study further, if any are found.

\newthought{Nuclear quantum effects} have been mentioned in the discussion of LiF in Sec.\,\ref{sec:results.experiments}. These are inherently absent in the formulation of classical Green Kubo theory, and correct treatment of quantum effects in dynamical properties such as heat transport poses a formidable challenge already on a conceptual level,~i.\,e.,~in the defintion of a heat flux estimator, and the correct evaluation of Kubo-transformed quantum mechanical correlation functions.
%\mscomment{see what the force field community has done}
%\FK{I cite the relevant literature in the next sentence. Explain more?}
However, some very recent path-integral molecular dynamics based approaches using classical force fields show some promising progress in the field~\cite{Luo.2020,Sutherland.2021}.

\section*{Perspective}
After discussing the status quo and potential futures of the methods used in this work, I want to discuss more long-term trends that are currently emerging in the field of heat transport simulations, and might have the power to push the topic forward considerably:
%In light of the constant development of solid state physics, materials science, and computational sciences, we see two main routes for the further improvement of the status quo in \emph{ab initio} heat transport simulations from equilibrium approaches: 
Green-Kubo simulations using machine-learned expressions for the potential-energy function $\mathcal{V} ({\bf R})$~\cite{Sosso.2012,Korotaev.2019,Qian.2019,Li.2020bb,Li.2020,Mangold.2020,Liu.2021,Han.2021,Fan.2021,Verdi.2021},
%\mscomment{is this all?}
%\FK{These are the available, citable approaches for GK from ML. Verdi/Kresse is still not even on arxiv.}
 and model Hamiltonian-based approaches using Boltzmann transport theory~\cite{Simoncelli.2019} or analytical Green's functions~\cite{Isaeva.2019,Dangic.2021}.

\subsection*{Machine learning Green Krubo}
\label{sec:outlook.ml}
The advent of machine-learned potentials with \emph{ab initio} quality~\cite{Lorenz.2004,Behler.2007,Bartok.2010,Bartok.2013,Shapeev.2016} coupled with on-the-fly or active-learning training strategies~\cite{Li.2015,Jinnouchi.2019,Podryabinkin.2017,Liu.2021} promises to become a versatile tool for dynamical simulations of materials in a range of subfields, especially for phenomena where the electronic structure of a given material is only of secondary importance,~i.\,e.,~in uncharged systems with electronically trivial defects. In such materials, machine-learned potentials can remove computational bottlenecks when aiming for statistical convergence in simulation time, as well as system and ensemble sizes, which in turn could make GK-based transport simulations of new materials accessible to more researchers than currently.

While several proofs of principle for GK simulations from machine-learned potentials exist already~\cite{Korotaev.2019,Li.2020,Mangold.2020}, some more fundamental problems persist,~e.\,g.,~the question of transferability across temperature and phase transitions in particular, or how long-range electrostatic interactions in polar systems can be properly described in these models~\cite{Artrith.2011,Grisafi.2019,Yue.2021,Kovacs.2021}. From materials discovery perspective, where one aims at studying a multitude of systems, it remains to be seen how straightforward and robust the parametrization of these force fields can be achieved in practice, and if the extra cost in human time in order to train the potentials can be reduced to a minimum without compromising their reliability. The aiGK data computed and made available in the course of this project can help to develop and benchmark these approaches.


\subsection*{Boltzmann transport and analytical Green's functions}
A complementary route for not too complex materials is the use of model Hamiltonians coupled with perturbation theory in the framework of lattice dynamics, which become increasingly sophisticated and continue to develop~\cite{Esfarjani.2008,Hellman.2013,Hellman.2013b,Errea.2014,Tadano.2018,Zhou.2019}. While these approaches are limited to non-diffusing materials with a definite long-range structural order and limited complexity,\footnote{See comments in Sec.\,\ref{sec:outlook.open_questions}.} they allow to study subtle dynamical effects such as the the behavior of thermal transport close to and across phase transitions with very good precision~\cite{Dangic.2021}, which is particularly important for potential thermoelectrics such as SnSe and GeTe~\cite{Zhao.2014,Dewandre.2016,Dangic.2021}. Further, these methods give straightforward access to nuclear quantum effects~\cite{Shulumba.2017}, which is a formidable task for molecular dynamics-based Green Kubo methods as discussed above.

Worth noting are more recent approaches that try to bridge the gap between state of the art Boltzmann transport~\cite{Simoncelli.2019} and analytical Green Kubo in terms of phonon Green's functions~\cite{Isaeva.2019,Dangic.2021}. Combined with self-consistent sampling techniques~\cite{Brown.2013}, these approaches promise to be a very efficient alternative for heat transport simulations in a wide class of solid systems.

Last but not least, lattice-dynamics techniques can be coupled with molecular dynamics simulations to extract fully anharmonic properties such as phonon lifetimes,~e.\,g.,~as we use them in the \emph{ab initio} Green Kubo method~\cite{Ladd.1986,Turney.2009,Zhang.2014,Carbogno.2016,Glensk.2019} -- the opportunities for combining these approaches are certainly not exhausted.



%\subsection{Exascale computing}
%We don't expect any novel physics being discovered through exascale computing.

%\subsection{Novel physics}
%At last, we would like to remark a few things about challenges from a physicist's point of view.
%
%\begin{itemize}
%	\item Methods:
%		\subitem ML potentials to make GK more feasible and accessible, remove computational bottleneck, go beyond GGA accuracy  
%		potentials:
%		\cite{Behler.2007,Bartok.2010,Bartok.2013,Shapeev.2016}
%		long-range:
%		\cite{Artrith.2011,Grisafi.2019,Yue.2021,Kovacs.2021}
%		active learning/on the fly:
%		\cite{Li.2015,Jinnouchi.2019,Podryabinkin.2017,Liu.2021}
%  	Shapeev: \cite{Korotaev.2019},	DeepotSE: \cite{Li.2020}, DonadioBehler: \cite{Mangold.2020}
%		\subitem analytical GK: crossover between GK and BTE approach for not-too-complex materials
%		\cite{Dangic.2021,Simoncelli.2019,Isaeva.2019}
%			\subsubitem NQE? \cite{Shulumba.2017,sutherland2021}
%		\subitem Question of reference positions, esp. close to phase transitions. Time-dependent?
%		\subitem 2D structures: People just do 3D transport, nobody really knows what's going on in 2D
%		\subitem Magnetism: No real theory for thermal transport in magnetic materials, only ad-hoc explanations like \cite{Stockem.2018}.
%		\subitem More generally: difficult systems for DFT, can be as banal as band-gap closing as observed for ZnSb
%	\item materials discovery
%		\subitem highlight that $\sigmaA$ is able to describe main trend in material space, improving on that by including more descriptors very well possible, will enable to accelerate novel materials discovery for thermal insulators
%			\CITE{Purcell?,Ramprasad,SISSO,Subgroup discovery,...}
%			\cite{Ouyang.2018,Goldsmith.2017,Chen.2019}
%		
%		\subitem Screening depends crucially on the choice of the screened materials,~i.\,e.,~materials not included in the screening cannot be found.
%		
%		\subitem only bulk systems considered \cite{Ferrando.2020}
%\end{itemize}
