\chapter{Bloch Theorem and Brillouin Zone}
\epigraph{\singlespacing \it ``The idea of periodicity in the reciprocal space is useless but, I think, harmless.''}{Paul Gartner}
\section{Bloch Theorem}
\label{sec:BlochTheorem}
The Schr\"odinger equation in 1d reads
\begin{align}
	\hat H \psi (x) = \left( - \frac{\nabla^2}{2m} + V(x) \right) \psi (x) = E \psi (x)~.
	\label{eq:app.bloch.se}
\end{align}
In a periodic potential,
\begin{align}
	V(x + a) = V(x)~,
	\label{eq:app.bloch.potential}
\end{align}
the periodicity can be expressed by stating that the translation operator $\hat T_a$ defined by its action,
\begin{align}
	\hat T_a f(x) = f(x + a)~,
	\label{eq:app.bloch.Ta}
\end{align}
commutes with the Hamiltonian,
\begin{align}
	\left[ \hat H , \hat T_a\right] = 0~.
	\label{eq:app.bloch.commute}
\end{align}
The eigenstates $\psi (x)$ of $\hat H$ are therefore also eigenstates of $\hat T_a$~\cite{Basdevant2000}. The translation operator is unitary, $\D{\hat T}_a = \hat{T}_a^{-1}$, but not hermitian. The eigenvalues $\lambda$ associated with $\hat T_a$ are thus complex numbers. By definition, one has \mbox{$\psi ( x + na ) = \lambda^n \psi(x)$}. Requiring bounded solutions, $\lim_{x \rightarrow \infty} \lvert \psi (x) \rvert < \infty$, imposes the condition $\lvert \lambda \rvert = 1$.
The function $\psi$ can therefore be written as
\begin{align}
	\psi (x) = c(x) u(x)~,
\end{align}
with a real, periodic function
\begin{align}
	u: \mathds R \rightarrow \mathds R
	\quad\text{with}\quad u(x + a) = u(x)~,
\end{align}
and a complex function of unit modulus,
\begin{align}
	c: \mathds R \rightarrow \mathds C
	\quad\text{with}\quad \left\lvert c(x) \right\rvert = 1~.
	\label{eq:app.bloch.c1}
\end{align}
We label each possible solution by the number $k$, then
\begin{align}
	c_k (x) = {\rm e}^{\im k x}
	%% don't impose uniqeness here
	%\quad\text{with}\quad k \in \left[0, \frac{2 \pi}{a} \right)
	\label{eq:app.bloch.c2}
\end{align}
% is a unique map from the domain $x \in [0, a)$ to the complex unit circle $\set{z \in \mathds C : \lvert z \rvert = 1}$. 
is a map from the domain $x \in \mathds R$ to the complex unit circle $\set{z \in \mathds C : \lvert z \rvert = 1}$. 
It then holds that $\hat T_a \psi_k (x) = {\rm e}^{\im k a} \psi(x)$,~i.\,e.,~$\psi_k$ is an eigenfunction of $\hat T_a$ with eigenvalue $\lambda = {\rm e}^{\im k a}$. We formulate the
\begin{thm}[Bloch]
	Solutions to the Schr\"odinger equation~\eqref{eq:app.bloch.se} with a periodic potential of periodicity $a$ are of the form
	\begin{align*}
		\psi_k (x) = {\rm e}^{\im k x} u_k (x)~,
	\end{align*}
	with a real, periodic function $u_k$.
%	 for each $k$ in the first Brillouin zone,
%	\begin{align*}
%		k \in \left[0, \frac{2 \pi}{a} \right)~.
%	\end{align*}
\end{thm}
The theorem is trivially extended to the 3d case by using the multiplication rule
\begin{align}
	\hat{T}_{{\bf a} + {\bf b}} f({\bf x}) = \hat{T}_{\bf a} \hat{T}_{\bf b} f({\bf x}) \equiv f({\bf x} + {\bf a} + {\bf b})~.
\end{align}
A more rigorous proof in terms of representation theory can be found,~e.\,g.,~in~\cite{Dresselhaus2007}.

\section{Brillouin Zone}
\label{sec:BrillouinZone}
We have not yet specified the range of the quantum number $k$. This can be done by requiring the complex function $c_k$ defined in Eq.\,\eqref{eq:app.bloch.c2} to map the interval $x \in [0, a)$ \emph{exactly once} to the unit circle so that $k$ is a \emph{unique} label for the eigenvalues ${\rm e}^{\im k a}$ of the translation operator $\hat T_a$.
We therefore define the
\begin{align}
	\text{Brillouin zone} = \set{k : k \in \left[ - \frac{\pi}{a}, \frac{\pi}{a}\right)}~.
\end{align}
For a wavefunction belonging to $k' = k + G$, where $G$ is an integer multiple of the the reciprocal lattice vector $b = 2\pi / a$, we would find
\begin{align}
	\hat T_a \psi_{k + G} (x) = {\rm e}^{\im k a} \psi_{k + G} (x)~.
\end{align}
They are therefore indistinguishable and we define $\psi_k$ and $\psi_{k+G}$ to be the same function,
\begin{align}
	\psi_k (x) = \psi_{k + G} (x)~.
\end{align}
This is sometimes termed ``periodicity of Bloch functions in reciprocal space''.

\chapter{Born-von Karman Supercell}
To ensure normalizability of the functions $\psi_{{\bf k}l}$, one additionally imposes the \emph{Born-von Karman boundary conditions}
\begin{align}
\psi_{{\bf k}l} ({\bf x} + {\bf A}_i) 
= \psi_{{\bf k}l} ({\bf x})
%\label{eq:dft.Bloch.4}
\end{align}
where each ${\bf A}_i$ is a linear combination of the primitive basis vectors $\set{{\bf a}_i}$,
\begin{align}
{\bf A}_i = S_i^{~j} {\bf a}_j\quad\text{with } S_i^{~j} \in \mathds Z~,
\end{align}
where $S$ is a non-singular matrix with integer elements. The space spanned by the $\set{{\bf A}_i}$ is parallelepiped of volume $V = N \, {\bf a}_1 \cdot ({\bf a}_2 \times {\bf a}_3)$, where $N = \det S$ is the number of unit cells that fit into the enlarged cell. This cell is therefore often called \emph{supercell}, and the matrix $S$ is denoted as a \emph{supercell matrix}.
With the Born-von Karman boundary conditions, the domain of all functions and functionals appearing in the Kohn-Sham equations become restricted to the supercell. The ideal, infinite crystal is obtained in the limit $N \rightarrow \infty$.
Using the periodic boundary condition expressed by Eq.\,\eqref{eq:dft.Bloch.4} in the Bloch functions given by Eq.\,\eqref{eq:dft.Bloch.2}, and the periodicity of the functions $u_{{\bf k} l}$, one finds that
\begin{align}
%	{\rm e}^{\im {\bf k} \cdot ({\bf x} + N_i {\bf a}_i)} u_{{\bf k} l} ({\bf x})
%		&= {\rm e}^{\im {\bf k} \cdot {\bf x}} u_{{\bf k} l} ({\bf x}) \nonumber \\
%	\implies
%		{\rm e}^{\im {\bf k} \cdot  N_i {\bf a}_i} 
%			&= 1 \nonumber \\
%	\implies
{\bf k} \cdot {\bf A}_i
&= 2 \pi m_i\quad\text{with } m_i \in \mathds N \text{ such that } 
\forall i: {\bf k} \cdot {\bf a}_i \leq 2 \pi~.
%\label{eq:dft.Bloch.5}
\end{align}
In total there are $N$ permissible values of $\bf k$ labelled by ${\bf m} = (m_1, m_2, m_3)$ that can be expressed in terms of the \emph{reciprocal lattice vectors}~\cite{Sands2002}
\begin{align}
{\bf B}^i 
= 2 \pi \varepsilon^{ijk} \frac{{\bf A}_j \times {\bf A}_k}{{\bf A}_1 \cdot ({\bf A}_2 \times {\bf A}_3)} ~,
%\label{eq:dft.Bloch.bi}
\end{align}
where $\varepsilon^{ijk}$ denotes the Levi-Civita symbol enforcing the correct ordering of $ijk$. The complete set of $\bf k$-values is
\begin{align}
{\bf k}_{\bf m} 
= \sum_{i=1}^3 m_i {\bf B}^i~.
%\label{eq:dft.Bloch.k_m}
\end{align}
The values of $\bf k$ given by Eq.\,\eqref{eq:dft.Bloch.k_m} are those sampled in real-space simulation in a box of the given size,~i.\,e.,~the \emph{Born-von Karman cell}.

\chapter{Linear Response Distribution Function}
\label{app:lr.f}
To solve for $\Delta f(t)$ defined in Eq.\,\eqref{eq:lr.df.2}, we introduce a shorthand notation such that
\begin{align}
\frac{\d \Delta f}{\d t} = -\im L \Delta f(t) - \im \Delta L (t) f^0~,
\label{eq:lr.df.3}
\end{align}
where the Liouville operator $L^0$ is defined by
\begin{align}
	\im L^0 g = \set{g, H^0}~,
	\label{eq:app.lr.L}
\end{align}
and similarly
\begin{align}
	\im \Delta L (t) g = \set{g, H'(t)}~.
	\label{eq:app.lr.L'}
\end{align}
Equation~\eqref{eq:lr.df.3} is a first order linear differential equation of the form
\begin{align}
	\frac{\d y}{\d t} + p(t) y = q(t)~,
	\label{eq:app.lr.dgl.1}
\end{align}
which is straightforward to solve by using an integrating factor as follows: We identify $y = \Delta f$, $p(t) = \im L^0$, and $q(t) = - \im \Delta L (t) f^0$. Following Ref.~\cite[p.\,68]{Lomen1986}, we define the integrating factor \mbox{$\rho (t) = \exp(\int \d t \, p (t)) = \exp (\im L^0 t)$}, multiply Eq.\,\eqref{eq:lr.dgl.1} with $\rho (t)$, and use that \mbox{$\frac{\d}{\d t} \, \rho(t) = \rho(t) p(t)$} to obtain
$$
\frac{\d}{\d t} (\rho (t) y) = \rho(t) q(t)~.
$$
This gets integrated to
$$
\rho(t) y = \int_{-\infty}^t \d t' \, \rho(t') q(t')
$$
under the boundary condition $y (t \to -\infty) = 0$. In total we obtain
\begin{align}
  y(t) 
    &= \rho^{-1} (t) \int_{-\infty}^t \d t' \, \rho(t') q(t')~, \\
  \implies
  \Delta f(t) 
    &= - {\rm e}^{- \im L^0 t}  \int_{-\infty}^t \d t' \, {\rm e}^{\im L^0 t'} \im \Delta L (t') f^0~.
\end{align}

\chapter{Explicit Formulas}
\section{Harmonic Approximation}
\label{app:ha.formulas}
In Sec.\,\ref{sec:dynmat.periodic}, we introduced the shorthand notation $s=(b, \b q)$, $-s=(b, -\b q)$ to write brief formulas. We give the explicit form of these formulas here.

\newthought{The normal mode coordinates} in the periodic case in terms of complex amplitudes $a^{(\dagger)}_b (\b q)$ read
\begin{subequations}
	\label{eq:u_b(q).amplitudes}
	\begin{align}
	u_b (\b q)
	&=   \frac{1}{\sqrt{2 \omega_b (\b q)}} \left[ \D a_b (- \b q) + \fD a_b (\b q)  \right] \\
	p_b (\b q)
	&= \im \sqrt{\frac{\omega_b (\b q)}{2}} \left[ \D a_b (- \b q) - \fD a_b (\b q)  \right]
	\end{align}
	\end{subequations}
	The inverse relation is given by
	\begin{subequations}
		\label{eq:a(q)}
		\begin{align}
		a_b (\b q)
		&= \sqrt{\frac{\omega_b (\b q)}{2}} u_b (\b q) + \frac{\im}{\sqrt{2 \omega_b (\b q)}} p_b (\b q) \\
		a^\dagger_b (-  \b q)
		&= \sqrt{\frac{\omega_b (\b q)}{2}} u_b (\b q) - \frac{\im}{\sqrt{2 \omega_b (\b q)}} p_b (\b q)
		\end{align}
		\end{subequations}
		The displacements are recovered by
		\begin{align}
		\b u_{i \b L}
		&= \frac{1}{\sqrt{N_{\b q}}} \sum_{b \b q} {\rm e}^{\im  \b q \cdot \b R^0_{i \b L}} \, \b e^\ast_{b i} (\b q) \fD u_b (\b q)
		\label{eq:u_iL}
		% \\
		% \b p_{i \b L}
		% &= \frac{1}{\sqrt{N_{\b q}}} \sum_{b \b q} {\rm e}^{\im  \b q \cdot \b R^0_{i \b L}} \, \b e^\ast_{b i} (\b q) \fD p_b (\b q)
		\end{align}
		%\end{subequations}
		and likewise for $\b p$.
		
		The Hamiltonian reads
		\begin{align}
		\mathcal H (u_b, p_b)
		&= \frac{1}{2} \sum_{b \b q} \left[ p^\ast_b (\b q) p_b (\b q) + \omega^2_b (\b q) u^\ast_b (\b q) u_b (\b q) \right] 
		\end{align}
		Equations of motion
		\begin{align}
		\ddot{u}_b (\b q)
		= \dot{p}_b (\b q)
		= - \frac{\partial \mathcal{H}}{\partial u_b^\ast (\b q)}
		\end{align}

\newpage

\section{Heat Capacity}
\begin{subequations}
\begin{align}
	\beta
		& = \frac{1}{k_{\rm B} T} \\
	c_V 
		&= \frac{\partial E}{\partial T} \\
	E (T)
		&= \sum_s \hbar \omega_s n_s (T) \\
	n_s (T)
		&= \frac{1}{e^{\beta \hbar \omega_s} - 1} \\
	\frac{\partial n_s}{\partial T}
		&= \frac{\hbar \omega_s}{k_{\rm B} T^2} \, n_s (n_s + 1) \\
	\implies c_V
		&= \sum_s \underset{c_{V, s}}{\underbrace{\frac{\hbar^2 \omega_s^2}{k_{\rm B} T^2} \, n_s (n_s + 1)}}
\end{align}
\end{subequations}
Classical limit $k_{\rm B} T \gg \hbar \omega_s$
\begin{subequations}
\begin{align}
	n_s (T) 
		&\to \frac{k_{\rm B} T}{\hbar \omega_s} \gg 1 \\
	\implies E(T)
		&\to 3 N k_{\rm B} T \\
	\implies c_V 
		&\to 3 N k_{\rm B}
\end{align}
\end{subequations}